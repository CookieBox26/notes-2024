\documentclass[b5paper,xelatex,ja=standard,11pt]{bxjsarticle}

% 余白を詰めてページ番号を下げる設定
\setpagelayout*{
  top=13truemm,
  bottom=18truemm,
  left=15truemm,
  right=15truemm}
\addtolength{\footskip}{5mm}

% 行間をやや詰める設定
\usepackage{leading}
\leading{14.5pt}
\usepackage{setspace}  % 目次の行間を詰める用

% 色名の読み込みと独自色定義
\usepackage[svgnames]{xcolor}
\definecolor{TextColor}{RGB}{51,51,51}  % #333333
\definecolor{PaleIris}{RGB}{229,227,246}
\definecolor{PaleGold}{RGB}{255,247,204}
\definecolor{PaleIris2}{RGB}{173,165,223}
\definecolor{PaleGold2}{RGB}{248,222,75}

% フォント定義 (日本語は小さめにスケール)
\setmainfont[
  Color=TextColor, BoldFont={[MPLUS1-ExtraBold.ttf]},
]{[MPLUS1-Regular.ttf]}  % 英数字本文
\setCJKmainfont[
  Scale=0.95, Color=TextColor, BoldFont={[MPLUS1-ExtraBold.ttf]},
]{[MPLUS1-Regular.ttf]}  % 日本語本文
\setCJKsansfont[
  Scale=0.95, Color=TextColor,
]{[MPLUS1-ExtraBold.ttf]}  % 日本語見出し
\setsansfont[
  Color=TextColor,
]{[MPLUS1-ExtraBold.ttf]}  % 英数字見出し
\setmonofont[
  Scale=1.05, %  Color=TextColor,
]{[MPLUS1Code-Regular.ttf]}  % 等幅

% 目次に点線を引く設定
\makeatletter
\renewcommand*\l@section{\@dottedtocline{1}{0.0em}{4.0em}}
\makeatother

% ソースコード表示設定
\usepackage{listings}
\usepackage[most,listings]{tcolorbox}
\tcbuselibrary{breakable}
\lstset{  % グローバル設定
  columns=fixed,  % 等幅
  basewidth=0.5em,  % 字間詰め
  lineskip=-3pt,  % 行間詰め
  basicstyle={\ttfamily\color{TextColor}},  % 全体フォント設定
  keywordstyle=[1]{\color{RoyalBlue}},  % kewords[1]の設定 (Python だと予約語)
  keywordstyle=[2]{\color{VioletRed}},  % kewords[2]の設定 (Python だと組み込み関数)
  stringstyle={\color{FireBrick}},  % 文字列リテラルの設定
  commentstyle={\color{SeaGreen}},  % コメントの設定
  showstringspaces=false,  % 半角スペース記号を表示しない (コピペの邪魔)
}

% 独自のコード枠の定義
\newtcolorbox{CODE}[2][]{
  enhanced, breakable,  % ページをまたぐのに必要
  arc=1pt, colframe=TextColor, boxrule=1pt,
  title={\addfontfeatures{Color=white}\addCJKfontfeatures{Color=white}\textbf{#2}},
  top=0pt, right=4pt, left=4pt, bottom=0pt,  % パディング
  #1
}

% 独自のセリフ枠の定義
\newtcolorbox{SERIFU}[4][]{
  enhanced,  % タイトルボックスの表示に必要
  breakable,  % ページをまたぐのに必要
  arc=7pt, top=7pt, right=7pt, left=7pt, bottom=7pt,  % パディング
  boxrule=0pt, frame hidden,  % フレーム隠蔽
  grow to left by=-31pt,  % キャラクター画像を表示するため左にマージン
  enlarge bottom by=-1pt, title={\,}, colback=#3, colbacktitle=#4,
  attach boxed title to top left={xshift=-34pt, yshift=-40pt},
  boxed title style={
    enhanced, arc=19pt, top=17pt, right=15pt, left=17pt, bottom=17pt,
    boxrule=0pt, frame hidden, underlay={\begin{tcbclipinterior}
      \includegraphics[width=39pt,keepaspectratio]{#2}\end{tcbclipinterior}},
  },
  #1
}

% ======================================================================
\begin{document}

\vspace*{0.05\textheight}
\centerline{\fontsize{27pt}{10pt}\selectfont\bf ほげふがほげふが}
\vspace*{10pt}
\centerline{\fontsize{15pt}{10pt}\selectfont\bf クッキー}
\vspace*{5pt}
\begin{center}
\begin{minipage}[t]{0.8\textwidth}
 こんにちは Hello こんにちは{\bf こんにちは}こんにちはこんにちは
こんにちはこんにちはこんにちはこんにちはこんにちはこんにちは。
\end{minipage}
\end{center}
\vspace*{3pt}

\addcontentsline{toc}{section}{参考文献}
\begin{thebibliography}{9}
  \bibitem{bib1} 参考文献。
\end{thebibliography}

\vspace*{-20pt}
\begin{spacing}{1.1}\textbf{\tableofcontents}\end{spacing}

\newcounter{mycounter}
\stepcounter{mycounter}
\newcommand*{\mysectiontitle}{第{\themycounter}話 \, ほげほげ}
\section*{\mysectiontitle}\addcontentsline{toc}{section}{\mysectiontitle}

\begin{SERIFU}[enlarge top by=3pt]{../images/a0_.png}{PaleIris}{PaleIris2}
こんにちは Hello こんにちは{\bf こんにちは}こんにちはこんにちは
こんにちはこんにちはこんにちはこんにちはこんにちはこんにちは。
こんにちは Hello こんにちは{\bf こんにちは}こんにちはこんにちは
こんにちはこんにちはこんにちはこんにちはこんにちはこんにちは。
こんにちは Hello こんにちは{\bf こんにちは}こんにちはこんにちは
こんにちはこんにちはこんにちはこんにちはこんにちはこんにちは。
\end{SERIFU}

\begin{SERIFU}[enlarge bottom by=0pt]{../images/b0_.png}{PaleGold}{PaleGold2}
こんにちは Hello こんにちは{\bf こんにちは}こんにちはこんにちは
こんにちはこんにちはこんにちはこんにちはこんにちはこんにちは。
こんにちは Hello こんにちは{\bf こんにちは}こんにちはこんにちは
こんにちはこんにちはこんにちはこんにちはこんにちはこんにちは。
こんにちは Hello こんにちは{\bf こんにちは}こんにちはこんにちは
こんにちはこんにちはこんにちはこんにちはこんにちはこんにちは。
こんにちは Hello こんにちは{\bf こんにちは}こんにちはこんにちは
こんにちはこんにちはこんにちはこんにちはこんにちはこんにちは。
こんにちは Hello こんにちは{\bf こんにちは}こんにちはこんにちは
こんにちはこんにちはこんにちはこんにちはこんにちはこんにちは。
\end{SERIFU}

\stepcounter{mycounter}
\renewcommand*{\mysectiontitle}{第{\themycounter}話 \, ふがふが}
\section*{\mysectiontitle}\addcontentsline{toc}{section}{\mysectiontitle}

こんにちは Hello こんにちは{\bf こんにちは}こんにちはこんにちは
こんにちはこんにちはこんにちはこんにちはこんにちはこんにちは。

\begin{CODE}[]{こんにちは Hello こんにちは}
\begin{lstlisting}[language=python]
import numpy as np
import pandas as pd
%matplotlib inline  
import matplotlib.pyplot as plt
import matplotlib.patches as patches
from matplotlib.colors import to_rgba
from matplotlib import font_manager
plt.rcParams['font.family'] = 'Ume Gothic'
plt.rcParams['font.size'] = 12
plt.rcParams['axes.linewidth'] = 1.5
plt.rcParams['grid.linewidth'] = 1.5
\end{lstlisting}
\end{CODE}

\stepcounter{mycounter}
\renewcommand*{\mysectiontitle}{第{\themycounter}話 \, ほげほげふがふが}
\section*{\mysectiontitle}\addcontentsline{toc}{section}{\mysectiontitle}

\begin{SERIFU}[enlarge top by=3pt, enlarge bottom by=5pt]
{../images/b0_.png}{PaleGold}{PaleGold2}
こんにちは Hello こんにちは{\bf こんにちは}こんにちはこんにちは
こんにちはこんにちはこんにちはこんにちはこんにちはこんにちは。
こんにちは Hello こんにちは{\bf こんにちは}こんにちはこんにちは
こんにちはこんにちはこんにちはこんにちはこんにちはこんにちは。
こんにちは Hello こんにちは{\bf こんにちは}こんにちはこんにちは
こんにちはこんにちはこんにちはこんにちはこんにちはこんにちは。
こんにちは Hello こんにちは{\bf こんにちは}こんにちはこんにちは
こんにちはこんにちはこんにちはこんにちはこんにちはこんにちは。
こんにちは Hello こんにちは{\bf こんにちは}こんにちはこんにちは
こんにちはこんにちはこんにちはこんにちはこんにちはこんにちは。
\end{SERIFU}

\rightline{\large\bf 終わり}

\end{document}
% ======================================================================
