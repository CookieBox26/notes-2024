\documentclass[b5paper,xelatex,ja=standard,11pt]{bxjsarticle}

% 余白を詰めてページ番号を下げる設定
\setpagelayout*{
  top=13truemm,
  bottom=18truemm,
  left=15truemm,
  right=15truemm}
\addtolength{\footskip}{5mm}

% 行間をやや詰める設定
\usepackage{leading}
\leading{14.5pt}
\usepackage{setspace}  % 目次の行間を詰める用

% 色名の読み込みと独自色定義
\usepackage[x11names]{xcolor}
\definecolor{TextColor}{RGB}{51,51,51}  % #333333
\definecolor{PaleIris}{RGB}{229,227,246}
\definecolor{PaleGold}{RGB}{255,247,204}
\definecolor{PaleIris2}{RGB}{173,165,223}
\definecolor{PaleGold2}{RGB}{248,222,75}

% 数式フォント用
\usepackage[mathrm=sym]{unicode-math}
\usepackage{amsmath}

% フォント定義 (日本語は小さめにスケール)
% 等幅フォントにはシンタクスハイライトを適用するために色を設定しない
% その代わりコード入りカラーボックスはページをまたげない
\setmainfont[Color=TextColor, BoldFont={MPLUS1-ExtraBold.ttf},]{MPLUS1-Regular.ttf}
\setCJKmainfont[Scale=0.95, Color=TextColor, BoldFont={MPLUS1-ExtraBold.ttf}]{MPLUS1-Regular.ttf}
\setCJKsansfont[Scale=0.95, Color=TextColor]{MPLUS1-ExtraBold.ttf}
\setsansfont[Color=TextColor]{MPLUS1-ExtraBold.ttf}
\setmonofont[Scale=1.0]{MPLUS1Code-Regular.ttf}  % 等幅
\setCJKmonofont[Scale=0.95]{MPLUS1Code-Regular.ttf}  % 日本語等幅

% 目次に点線を引く設定
\makeatletter
\renewcommand*\l@section{\@dottedtocline{1}{0.0em}{4.0em}}
\makeatother

% ソースコード表示設定
\usepackage{listings}
\usepackage[most,listings]{tcolorbox}
%\tcbuselibrary{breakable}
\lstset{  % グローバル設定
  columns=fixed,  % 等幅
  basewidth=0.5em,  % 字間詰め
  lineskip=-3pt,  % 行間詰め
  basicstyle={\ttfamily\color{TextColor}},  % 全体フォント設定
  keywordstyle=[1]{\color{DodgerBlue3}},  % kewords[1]の設定 (Python だと予約語)
  keywordstyle=[2]{\color{VioletRed3}},  % kewords[2]の設定 (Python だと組み込み関数)
  stringstyle={\color{Firebrick3}},  % 文字列リテラルの設定
  commentstyle={\color{SeaGreen4}},  % コメントの設定
  showstringspaces=false,  % 半角スペース記号を表示しない (コピペの邪魔)
}

% 独自のセリフ用カラーボックスの定義
% 共通設定
\tcbset{
  serifubox/.style={
    enhanced,  % タイトルボックスの表示に必要
    breakable,  % ページをまたぐのに必要
    arc=7pt, top=7pt, right=7pt, left=7pt, bottom=7pt,  % パディング
    boxrule=0pt, frame hidden,  % フレーム隠蔽
    grow to left by=-31pt,  % キャラクター画像を表示するため左にマージン
    enlarge bottom by=-1pt, title={\,},
    attach boxed title to top left={xshift=-34pt, yshift=-40pt},
    boxed title style={
      enhanced, arc=19pt, top=17pt, right=15pt, left=17pt, bottom=17pt,
      boxrule=0pt, frame hidden, underlay={\begin{tcbclipinterior}
        \includegraphics[width=39pt,keepaspectratio]{#1}\end{tcbclipinterior}},
    }
  }
}
% 部長
\newtcolorbox{SERIFU1}[1][]{
  serifubox={../images/a0_.png},
  colback=PaleIris, colbacktitle=PaleIris2, #1
}
% 副部長
\newtcolorbox{SERIFU2}[1][]{
  serifubox={../images/b0_.png},
  colback=PaleGold, colbacktitle=PaleGold2, #1
}

% 独自の汎用カラーボックスの定義
% 共通設定
\tcbset{
  custombox/.style={
    enhanced,  % タイトルボックスの表示に必要
    breakable,  % ページをまたぐのに必要
    arc=2pt, colframe=TextColor, boxrule=1pt, colback=Snow1,
    right=7pt, left=7pt,
    title={\addfontfeatures{Color=white}\addCJKfontfeatures{Color=white}\textbf{#1}}
  }
}
% 独自のコード用カラーボックスの定義
\newtcolorbox{CODE}[2][]{
  custombox={#2}, top=0pt, bottom=0pt, #1
}
% 独自の命題枠の定義
\newtcolorbox{PROP}[2][]{
  custombox={#2}, top=7pt, bottom=7pt, #1
}

% ハイパーリンク
\PassOptionsToPackage{hyphens}{url}
\usepackage[colorlinks=true]{hyperref}
\hypersetup{urlcolor=SteelBlue3}

% ======================================================================
\begin{document}

\vspace*{0.05\textheight}
\centerline{\fontsize{27pt}{10pt}\selectfont\bf クッキーの雑記\rule[-0.5em]{0pt}{0pt}}
\centerline{\fontsize{13pt}{10pt}\selectfont\bf クッキー\footnote{
お気付きの点がありましたら、大変お手数ですが cookie-box[at]cookie-box.info までご連絡ください。
}\rule[-0.5em]{0pt}{0pt}}
\centerline{\fontsize{13pt}{10pt}\selectfont\bf May 5, 2024 \, \, \rule[-0.5em]{0pt}{0pt}}
\begin{center}
\begin{minipage}[t]{0.8\textwidth}
 こんにちは Hello こんにちは{\bf こんにちは}こんにちはこんにちは
こんにちはこんにちはこんにちはこんにちはこんにちはこんにちは。
\end{minipage}
\end{center}
\vspace*{3pt}

\addcontentsline{toc}{section}{参考文献}
\begin{thebibliography}{9}
  \bibitem{bib1} 参考文献。
\end{thebibliography}

\vspace*{-20pt}
\begin{spacing}{1.1}\textbf{\tableofcontents}\end{spacing}

\newcounter{mycounter}
\stepcounter{mycounter}
\newcommand*{\mysectiontitle}{第{\themycounter}話 \, ほげほげ}
\section*{\mysectiontitle}\addcontentsline{toc}{section}{\mysectiontitle}

\begin{SERIFU1}[enlarge top by=3pt]
こんにちは Hello こんにちは{\bf こんにちは}こんにちはこんにちは
こんにちはこんにちはこんにちはこんにちはこんにちはこんにちは。
こんにちは Hello こんにちは{\bf こんにちは}こんにちはこんにちは
こんにちはこんにちはこんにちはこんにちはこんにちはこんにちは。
こんにちは Hello こんにちは{\bf こんにちは}こんにちはこんにちは
こんにちはこんにちはこんにちはこんにちはこんにちはこんにちは。
\end{SERIFU1}

\begin{SERIFU2}
こんにちは Hello こんにちは{\bf こんにちは}こんにちはこんにちは
こんにちはこんにちはこんにちはこんにちはこんにちはこんにちは。
こんにちは Hello こんにちは{\bf こんにちは}こんにちはこんにちは
こんにちはこんにちはこんにちはこんにちはこんにちはこんにちは。
こんにちは Hello こんにちは{\bf こんにちは}こんにちはこんにちは
こんにちはこんにちはこんにちはこんにちはこんにちはこんにちは。
こんにちは Hello こんにちは{\bf こんにちは}こんにちはこんにちは
こんにちはこんにちはこんにちはこんにちはこんにちはこんにちは。
こんにちは Hello こんにちは{\bf こんにちは}こんにちはこんにちは
こんにちはこんにちはこんにちはこんにちはこんにちはこんにちは。
\end{SERIFU2}

\stepcounter{mycounter}
\renewcommand*{\mysectiontitle}{第{\themycounter}話 \, ふがふが}
\section*{\mysectiontitle}\addcontentsline{toc}{section}{\mysectiontitle}

こんにちは Hello こんにちは{\bf こんにちは}こんにちはこんにちは
こんにちはこんにちはこんにちはこんにちはこんにちはこんにちは。

\begin{CODE}[]{こんにちは Hello こんにちは}
\begin{lstlisting}[language=python]
import numpy as np
import pandas as pd
%matplotlib inline  
import matplotlib.pyplot as plt
import matplotlib.patches as patches
from matplotlib.colors import to_rgba
from matplotlib import font_manager
plt.rcParams['font.family'] = 'Ume Gothic'
plt.rcParams['font.size'] = 12
plt.rcParams['axes.linewidth'] = 1.5
plt.rcParams['grid.linewidth'] = 1.5
\end{lstlisting}
\end{CODE}

\stepcounter{mycounter}
\renewcommand*{\mysectiontitle}{第{\themycounter}話 \, ほげほげふがふが}
\section*{\mysectiontitle}\addcontentsline{toc}{section}{\mysectiontitle}

\begin{SERIFU1}[enlarge top by=3pt, enlarge bottom by=5pt]
こんにちは Hello こんにちは{\bf こんにちは}こんにちはこんにちは
こんにちはこんにちはこんにちはこんにちはこんにちはこんにちは。
\end{SERIFU1}

\begin{PROP}{定理 0.1.0 〈\, 制約がないときの1次の必要条件 \,〉}
$x^\ast$ を問題(P)の局所最適解とする。但し、$f$ は $x^\ast$ で微分可能であるものとする。
 (P) $\displaystyle \underset{x \in \mathbb{R}^n}{\mathrm{minimize}} \; \; f(x)$
このとき,以下を満たす $s \in \mathbb{R}^n$ は存在しない。
\begin{eqnarray*}
\nabla f(x^\ast)^\top s < 0
\end{eqnarray*}
\end{PROP}

\rightline{\large\bf 終わり}

\end{document}
% ======================================================================
